\documentclass{article}

% Language setting
% Replace `english' with e.g. `spanish' to change the document language
\usepackage[italian]{babel}

% Set page size and margins
% Replace `letterpaper' with`a4paper' for UK/EU standard size
\usepackage[letterpaper,top=2cm,bottom=2cm,left=3cm,right=3cm,marginparwidth=1.75cm]{geometry}

% Useful packages
\usepackage{amsmath}
\usepackage{graphicx}
\usepackage[colorlinks=true, allcolors=blue]{hyperref}

\title{Progetto di Reti Logiche}
\author{
  William Zeni\\
  \texttt{matricola 10613915}
  \and
  Cristina Urso\\
  \texttt{matricola 10599689}
}
\renewcommand{\abstractname}{}


\begin{document}
\maketitle

\begin{abstract}
  \centering
  Progetto sostenuto presso il Politecnico di Milano, diretto dal professor Gianluca Palermo nell'anno 2021.
\end{abstract}

\section{introduction}
\subsection{Scopo del progetto}
Write somenthing here

\subsection{Specifiche generali}
Write somenthing here

\subsection{Interfaccia del componente}
Write somenthing here

\subsection{Dati e Descrizione memoria}
Write somenthing here

\section{Desing Pattern}
\subsection{Scelte Progettuali}
Write somenthing here

\subsection{Descrizione degli Stati}
\subsubsection{START}
Write somenthing here

\subsubsection{INIT}
Write somenthing here

\subsubsection{ABILIT READ}
Write somenthing here

\subsubsection{ABILIT WRITE}
Write somenthing here

\subsubsection{WAIT MEM}
Write somenthing here

\subsubsection{GET RC}
Write somenthing here

\subsubsection{GET DIM}
Write somenthing here

\subsubsection{READ PIXEL}
Write somenthing here

\subsubsection{GET MINMAX}
Write somenthing here

\subsubsection{GET DELTA}
Write somenthing here

\subsubsection{CALC SHIFT}
Write somenthing here

\subsubsection{GET PIXEL}
Write somenthing here

\subsubsection{CALC NEWPIXEL}
Write somenthing here

\subsubsection{WRITE PIXEL}
Write somenthing here

\subsubsection{DONE}
Write somenthing here

\subsubsection{WAITINGPIC}
Write somenthing here

\section{Risultati dei Test}
Write somenthing here!

\section{Conclusioni}
Write somenthing here

\end{document}
