\documentclass{article}

% Language setting
% Replace `english' with e.g. `spanish' to change the document language
\usepackage[italian]{babel}

% Set page size and margins
% Replace `letterpaper' with`a4paper' for UK/EU standard size
\usepackage[letterpaper,top=2cm,bottom=2cm,left=3cm,right=3cm,marginparwidth=1.75cm]{geometry}

% Useful packages
\usepackage{amsmath}
\usepackage{graphicx}
\usepackage[colorlinks=true, allcolors=blue]{hyperref}

\title{Progetto di Reti Logiche}
\author{
  William Zeni\\
  \texttt{matricola 10613915}
  \and
  Cristina Urso\\
  \texttt{matricola 10599689}
}
\renewcommand{\abstractname}{}


\begin{document}
\maketitle

\begin{abstract}
  \centering
  Progetto sostenuto presso il Politecnico di Milano, diretto dal professor Gianluca Palermo nell'anno 2021.
\end{abstract}

\section{introduction}
\subsection{Scopo del progetto}
Write somenthing here

\subsection{Specifiche generali}
Write somenthing here

\subsection{Interfaccia del componente}
Write somenthing here

\subsection{Dati e Descrizione memoria}
Write somenthing here

\section{Desing Pattern}
\subsection{Scelte Progettuali}
Write somenthing here

\subsection{Descrizione degli Stati}
\subsubsection{START}
Lo stato di \texttt{START} è stato pensato come stato di attesa iniziale. Questo stato viene invocato in due situazioni differenti: se il segnale di \texttt{i\_rst} viene portato alto, oppure quando il segnale \texttt{i\_start} viene riportato basso dopo la computazione di un immagine. Lo stato \texttt{START} rimane tale fino a quando il segnale \texttt{i\_start} non venga posto alto. In quel momento lo stato successivo viene impostato \texttt{INIT}.

\subsubsection{INIT}
Lo stato \texttt{INIT} è uno stato di transizione nel quale il processore si assicura che i segnali siano inizializzati con i valori opportuni. Successivamente imposta lo stato prossimo a \texttt{ABILIT READ}.

\subsubsection{ABILIT READ}
Lo stato \texttt{ABILIT READ} è lo stato attraverso il quale abilitiamo la memoria alla sola lettura. Viene richiamato in momenti diversi del progetto ed in base allo stato chiamante, instrada lo stato prossimo a quello opportuno.

\subsubsection{ABILIT WRITE}
Lo stato \texttt{ABILIT WRITE} abilita la memoria alla lettura e alla scrittura. Viene invocato subito dopo aver computato il valore del nuovo pixel e in nessun altro momento. Instrada poi lo stato prossimo a \texttt{WRITE PIXEL}.

\subsubsection{WAIT MEM}
Lo stato \texttt{WAIT MEM} è uno stato centrale durante la gestione del flusso di dati. Sostanzialmente "spreca" un ciclo di clock. Questo ci assicura sia in caso di scrittura, sia in caso di lettura che i segnali in ingress e in uscita siano letti o scritti correttamente. Nel caso specifico alla quale ci rifacciamo, alcune chiamate a questo stato potevano essere evitate. Questa informazione è emersa durante lo stress test a cui il processore è stato sottoposto. Tuttavia, abbiamo preferito lasciarle per mantenere la stuttura del processore. Ciò, a nostro avviso, permette una maggior robustezza, sebbene una aumento nella latenza della computazione.

\subsubsection{GET RC}
Write somenthing here

\subsubsection{GET DIM}
Write somenthing here

\subsubsection{READ PIXEL}
Write somenthing here

\subsubsection{GET MINMAX}
Write somenthing here

\subsubsection{GET DELTA}
Write somenthing here

\subsubsection{CALC SHIFT}
Write somenthing here

\subsubsection{GET PIXEL}
Write somenthing here

\subsubsection{CALC NEWPIXEL}
Write somenthing here

\subsubsection{WRITE PIXEL}
Write somenthing here

\subsubsection{DONE}
Write somenthing here

\subsubsection{WAITINGPIC}
Write somenthing here

\section{Risultati dei Test}
Write somenthing here!

\section{Conclusioni}
Write somenthing here

\end{document}
